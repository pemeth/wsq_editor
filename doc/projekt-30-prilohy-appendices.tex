% Tento soubor nahraďte vlastním souborem s přílohami (nadpisy níže jsou pouze pro příklad)

\chapter{Inštalácia a spustenie aplikácie} \label{sec:instalacia}
  Aplikácia je určená ako voľne distribuovateľná spolu so zdrojovými súbormi. Prerekvizitami aplikácie sú Python3 a príslušný správca balíčkov \texttt{pip},
  pomocou ktorých je možné doinštalovať všetky potrebné balíčky. Tieto balíčky sú uvedené v súbore \texttt{requirements.txt}. Pred inštaláciou odporúčam
  vytvoriť virtuálne prostredie pre Python pomocou modulu \texttt{venv}. Odporúčaný postup inštalácie je nasledovný:
  \begin{enumerate}
    \item Vytvorenie virtuálneho prostredia \texttt{venv} pomocou príkazu \texttt{python3 -m venv <cesta pre novy priecinok prostredia>}.
    \item Aktivácia prostredia pomocou skriptu \texttt{cesta/k/venv/bin/activate}. Na sytémoch Windows je táto cesta \texttt{cesta/k/venv/Scripts/activate}.
    \item Inštalácia balíčkov pomocou príkazu \texttt{python -m pip install -r requirements.txt}.
  \end{enumerate}
  Na systémoch \emph{Linux} a \emph{macOS} sa tieto príkazy vykonávajú v príkazovom riadku terminálu. Na systémoch \emph{Windows} v prostredí PowerShell.
  Využitie virtuálneho prostredia je plne voliteľné a je sem zahrnuté ako možnosť pre prípad, že užívateľ nechce nainštalovať potrebné balíčky do systémovej
  verzie Python\footnote{Po odstránení priečinku virtuálneho prostredia sa vymažú všetky v rámci neho inštalované balíčky.}.

  Aplikácia sa vo virtuálnom prostredí spúšťa príkazom \texttt{python app.py}.

% Umístění obsahu paměťového média do příloh je vhodné konzultovat s vedoucím
%\chapter{Obsah přiloženého paměťového média}

%\chapter{Manuál}

%\chapter{Konfigurační soubor}

%\chapter{RelaxNG Schéma konfiguračního souboru}

%\chapter{Plakát}

% //TODO prilohy